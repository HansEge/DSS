%!TEX root = ../../Main.tex
\graphicspath{{Chapters/Indledning/}}
%-------------------------------------------------------------------------------

\chapter{Simple linear regression / Multiple linear regression}

The simple Linear Regression approach is a quick and simple method for fitting a line through a 2-dimensional dataset. It is assumed that there is a approximately linear relationship between the two dimensions. This can be written mathematically as:

\begin{equation} \label{eq:Linear_Regression_eq}
	\begin{split}
		Y & \approx \beta_0 + \beta_1 * X
	\end{split}
\end{equation} 

\cref{eq:Linear_Regression_eq} can also be seen as "Regressing Y onto X". As an example the dataset Advertising.csv contains sales og a certain product and advertisement money spent on certain media platforms. X represents TV advertising and Y represents sales. It is the possible to regress sales onto TV. 

In order to do this, we need to calculate the constants $\beta_0$ and $\beta_1$ which represents the intercept and slope terms in the linear model. 