%!TEX root = ../Main.tex

\chapter{Support vector machines}
In this chapter we’ll discuss the subject Support Vector Machines (SVM). SVM performs well in a variety of settings and are often considered one of the best “out of the box “classifiers.  In this chapter we’ll touch upon two different variants of SVM. First, the simple an intuitive classifier called maximal margin classifier, which the support vector machine is a generalization of. Although the \textit{Maximal Margin Classifier} is elegant and simple, there is a lot of datasets where the classifier can’t be applied, because it requires the classes to be separated by a linear boundary. So, we’ll also touch upon an extension of the \textit{Maximal Margin Classifier} called the \textit{Support Vector Classifier}, which can be applied to a broader range of datasets. 

\section{Maximal Margin Classifier}
Maximal Margin Classifier is the simplest of different type of SVM. It tries to find a plane that separates the classes in the feature space. When we talk about a plane to separate the classes, we talk about a hyperplane. 

A hyperplane in \textit{p} dimensions is a flat affine subspace of dimensions \textit{p}-1. If we’ll look at two dimensions, a hyperplane would be a one-dimensional subspace. If we look at \textit{p} > 3 dimensions, it can be hard to visualize the hyperplane. \\
A hyper plane has the form in p-dimensional settings \cite{book_2015}: 
\begin{equation}
\beta_0 + \beta_1 X_1 + \beta_2 X_2 +...+ \beta_p X_p = 0\label{eq:hyperplane}
\end{equation}
If a point $X = (X_1 , X_2 ,..., X_p)^T$ satisfies \cref{eq:hyperplane} then X lies on the hyperplane. If on the other hand X does not satisfy the \cref{eq:hyperplane} like: 
\begin{equation}
\beta_0 + \beta_1 X_1 + \beta_2 X_2 +...+ \beta_p X_p < 0\label{ eq:hyperplane1 }
\end{equation}
It tells us that X lies to one side of the hyperplane, and if its greater than zero, then it lies on the other side of the hyperplane. So, one can easily determine on which side of the hyperplane a point lies. This fact makes it possible to use a hyperplane to classify variables. 
Suppose we have a n x p matrix X that consists of n training observations in p-dimensional space, and these observations can fall into two classes depending on the outcome of  \cref{eq:hyperplane}, which we represent as \{-1,1\} where -1 represent the one class and 1 represents the other class. 
Like other classifiers our goal is to develop a classifier that based on the training data can correctly classify the test data. 

On \cref{fig:Hyperplanes} is illustrated three different hyperplanes (the green lines) that separate the training data correctly. 
\begin{figure}[H]
	\centering
	\includegraphics[width=5cm]{Img/Hyperplanes.PNG}
	\caption{Illustrates different hyperplanes to separate the classes}
	\label{fig:Hyperplanes}
\end{figure} 

If separating hyperplanes like these exist, it is possible to a very natural classifier, where a test observation is at assigned to a class depending on which side of the hyperplane its located. Then we classify the test observation based on the sign of $f(x^*) = \beta_0 + \beta_1 x_1^* + \beta_2 x_2^* + ... + \beta_p x_p^*$. If its positive then its assigned to class 1, and if its negative, then assigned to class -1. The magnitude of $f(x^*)$ tells us how far form the hyperplane the test variable is, which also tells us how certain we are about a class assignment. 

If there exist a hyperplane that perfectly separate out training data, that means there exist an infinite number of hyperplanes that perfectly separates the training data. And how to chose between this infinite number of hyperplanes to use as the classifier? This is where the Maximal Margin Classifier comes into the picture. This is selecting the hyperplane that is the farthest from the training data. This is done by calculating the perpendicular distance from each training observation.  The smallest distance from the training observation to the hyperplane is called the margin. \\
So, the aim is to generate a hyperplane that has the largest margin on the training set, and hop it also has a large margin on the test set. On \cref{fig:Hyperplane_margin} is illustrated the hyperplane using the Maximal Margin Classifier.

\begin{figure}[H]
	\centering
	\includegraphics[width=5cm]{Img/Hyperplane_margin.PNG}
	\caption{Hyperplane maximum margin}
	\label{fig:Hyperplane_margin}
\end{figure} 
The maximum hyperplane can be solved as follows\cite{book_2015}: 
$$\underset{\beta_0 ,\beta_1 ,...,\beta_p}{\text{Maximize M}}$$

$$subject\: to\:\sum\limits_{j=1}^{p} \beta_j^2 = 1$$

$$y_i(\beta_0 + \beta_1 x_i1 + \beta_2 x_i2 + ... + \beta_p x_ip)  \geq  M\: \forall\: i=1,...,n$$

\section{Support Vector Classifier}
The maximal margin classifier is a natural way to perform classification, but it has its drawback. As already explained often no separation hyperplane exists, and so there is no maximal margin classifier. This leads us an extension of maximal margin classifier also called Support Vector Classifier.  
The support vector classifier introduces something called “soft margin”. A soft margin, does not seek the largest possible margin so every observation is on the correct side of the margin and hyperplane. Instead it allows some observations to be on the incorrect side of the margin, or even incorrect side of the hyperplane. This is also very useful when you have a lot of noisy data, because then one observation can change the separating hyperplane drastically and often not for the better.  
This actually has to be the case when there doesn’t exist a hyperplane that perfectly separates the observations. 

The Support Vector Classifier classifies a test observation like the Maximum Margin Classifier, on which side of the hyperplane it lies. But instead, of perfectly separate all training data, the Support Vector Classifier classifies most of the training data into the correct classes and may misclassify a few of the training data observations. The solution to this problem can be written as follows: 

$$\underset{\beta_0 ,\beta_1 ,...,\beta_p, \epsilon_1,...,\epsilon_n}{\text{Maximize M}}$$

$$subject\: to\:\sum\limits_{j=1}^{p} \beta_j^2 = 1$$

$$y_i(\beta_0 + \beta_1 x_i1 + \beta_2 x_i2 + ... + \beta_p x_ip)  \geq  M(1-\epsilon_i)$$

$$\epsilon_i \geq, \sum_{i=1}^{n}\epsilon_i \leq C$$

Where C is a nonnegative tuning parameter. As before M is the width of the margin. The new parameter $\epsilon $ are a slack variable, that allow individual observations to be on the wrong side of the hyperplane. We still classify test observations based on the sign of $f(x^*) = \beta_0 + \beta_1 x_1^* + \beta_2 x_2^* + ... + \beta_p x_p^*$. \\
The slack variable $\epsilon_i $ tells where the \textit{i}th observation is located relative to the hyperplane and margin. If $\epsilon_i = 0$ then the observation is on the correct side of the margin. If $\epsilon_i > 0$ then its on the wrong side of the margin, and if $\epsilon_i > 1$ then its on the wrong side of the hyperplane. 

C is treated as a tuning parameter, that controls the bias-variance trade-off. When C is small, we seek narrow margins that are rarely violated, which may have low bias but high variance. If C is large, the margin is wider and allow more violations; this amount to fitting the data less hard which potentially gives more bias but may have lower variance. 

An example of different C values can be seen from Lab 9.6.1 (The source code can be seen in the appendix). In this Lab exercise 20 random training values has been generated and 10 of them was assigned with the value 1 and 10 with the value -1. We then applied the support vector classifier on the data with the C value of 1 and 0.1, and then plotted the results. On \cref{fig:SVC_C_values} it is clear that the hyperplane and the margin changes with the value of C. As explained the smaller the C value is the wider is the margin. 

\begin{figure}[H]
	\centering
	\includegraphics[width=\textwidth]{Img/SVC_C_values.PNG}
	\caption{Lab 9.6.1 result of different C values}
	\label{fig:SVC_C_values}
\end{figure} 
