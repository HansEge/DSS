%!TEX root = ../../Main.tex
%!TeX spellcheck = en_US
\graphicspath{{Chapters/Indledning/}}
%-------------------------------------------------------------------------------

\chapter{Classification}

Classification refers to the practice of predicting qualitative responses. While there are many different classification techniques (also known as classifiers), In this chapter we will touch upon two of the most widely-used ones.

\section{Logistic regression}

Linear regression can be a very useful tool in field of supervised learning, however it requires assumptions that the responds variable $Y$ is quantitative, which often isn't the case. In many contexts, like when discussing eye color, the variable is qualitative, taking on values such blue, green, or brown.

\subsection{Linear vs. Logistic regression}

Why is linearly regression ill suited for cases involving a qualitative responds?


\subsection{Maximum likelihood}

\subsection{Predicting}

\section{Linear discriminant analysis}